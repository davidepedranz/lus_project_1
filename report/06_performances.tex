\section{Performances}
\label{sec:performances}

In this section, we present the performances of the models presented in \cref{sec:implementation} and \ref{sec:improvements}.

\subsection{Baseline}
We use the first model with a unigram and no smoothing a a baseline to evaluate the performances of the different n-gram orders and smoothing methods.
The model with these parameters assign to each word the concept with the highest frequency in the training set for the specific word.
The baseline model reached a F1 of $57.13\%$, as showed in \cref{tab:v1-ngrams}. 

\begin{table}
	\centering
    \begin{tabular}{ l l l l l }
    	\toprule
    		n & smoothing & prec. & recall & F1 \\
    	\midrule
            1 & unsmoothed & 54.64\% & 59.85\% & 57.13\% \\
2 & unsmoothed & 78.39\% & 74.15\% & 76.21\% \\
2 & witten\_bell & 78.51\% & 74.34\% & 76.37\% \\
3 & unsmoothed & 76.55\% & 74.52\% & 75.52\% \\
3 & witten\_bell & 76.58\% & 74.61\% & 75.58\% \\
4 & unsmoothed & 77.15\% & 74.89\% & 76.00\% \\
4 & witten\_bell & 77.11\% & 75.34\% & 76.22\% \\
5 & unsmoothed & 77.35\% & 74.79\% & 76.05\% \\
5 & witten\_bell & 77.03\% & 75.62\% & 76.32\% \\
6 & unsmoothed & 76.72\% & 74.61\% & 75.65\% \\
6 & witten\_bell & 74.35\% & 73.05\% & 73.69\% \\

    	\bottomrule
	\end{tabular}
    \caption{Comparison of the performances of different n-gram orders for model 1.}
	\label{tab:v1-ngrams}
\end{table}

\subsection{n-gram Order}
The model without smoothing has the best performances for bigrams and 5-grams, reaching respectively a F1 of $76.21\%$  and $76.05\%$.
The smoothing helps to get slightly better performances, but does not change the ranking of the n-gram orders.
All models with n-gram order greater then $1$ perform significantly better than the baseline, with an increase in the F1 score of about $20\%$.

\subsection{Smoothing Methods}

\subsection{Features}
