\section{Data Set}
\label{sec:dataset}

\begin{table}[t!]
	\centering
    \begin{tabular}{ l l l }
    	\toprule
    		\multicolumn{1}{l}{concept} & \multicolumn{1}{l}{train} & \multicolumn{1}{c}{test} \\
    	\midrule
            actor.name & 437 & 157 \\
actor.nationality & 6 & 1 \\
actor.type & 3 & 2 \\
award.category & 1 & 4 \\
award.ceremony & 13 & 7 \\
character.name & 97 & 21 \\
country.name & 212 & 67 \\
director.name & 455 & 156 \\
director.nationality & 2 & 1 \\
movie.description & 2 & 0 \\
movie.genre & 98 & 37 \\
movie.gross\_revenue & 34 & 20 \\
movie.language & 207 & 72 \\
movie.location & 21 & 11 \\
movie.name & 3157 & 1030 \\
movie.release\_date & 201 & 70 \\
movie.release\_region & 10 & 6 \\
movie.star\_rating & 1 & 1 \\
movie.subject & 247 & 59 \\
movie.type & 0 & 4 \\
person.name & 280 & 66 \\
person.nationality & 2 & 0 \\
producer.name & 336 & 121 \\
rating.name & 240 & 69 \\

    	\bottomrule
	\end{tabular}
    \caption{Frequency of the concepts in the data set.}
	\label{tab:frequencies}
\end{table}

The data set we used is NL-SPARQL, a corpora of sentences from the movie domain.
For each word in the sentences, the \ac{POS} and the word stem are provided as additional features.
The aim is to tag the sentences with the concepts, like the actor name or the movie release date.

\subsection{Corpora}
The corpora is already divided in the train and test subsets.
The train subset consists of $\input{counts/train.sentences}$ sentences,
for a total of $\input{counts/train.tokens}$ tokens.
The test subset consists of $\input{counts/train.sentences}$ sentences,
for a total of $\input{counts/train.tokens}$ tokens.
\cref{tab:frequencies} show the distribution of the concepts in the corpora.
We can notice that some concepts, like \texttt{director.nationality} and \texttt{award.category},
are very rare in the train set.
The concept \texttt{movie.type} is present only in the test set, so it is very hard for any model to predict it.

\subsection{IOB notation}
The concepts are reported in the \ac{IOB} format.
Each concept can spanw on multiple contingent words:
the \texttt{B-} prefix indicates that the tag is the beginning of a chunk,
the \texttt{I-} prefix indicates that the tag is inside a chunk.
A chunk is delimited by either a \texttt{O} or a \texttt{B-} tag.
The \texttt{O} tag indicates that no concept has been assigned to the given token.

\subsection{Evalutation}
The performances of the models are evaluated using the \texttt{conlleval.pl}
script\footnote{\url{http://www.cnts.ua.ac.be/conll2000/chunking/output.html}}.
The script computes precision, recall and F1 score for each concept and for the whole test set.
