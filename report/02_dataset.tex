\section{Data Set}
\label{sec:dataset}

\begin{table}[t!]
	\centering
    \begin{tabular}{ l l l }
    	\toprule
    		\multicolumn{1}{l}{concept} & \multicolumn{1}{l}{train} & \multicolumn{1}{c}{test} \\
    	\midrule
            actor.name & 437 & 157 \\
actor.nationality & 6 & 1 \\
actor.type & 3 & 2 \\
award.category & 1 & 4 \\
award.ceremony & 13 & 7 \\
character.name & 97 & 21 \\
country.name & 212 & 67 \\
director.name & 455 & 156 \\
director.nationality & 2 & 1 \\
movie.description & 2 & 0 \\
movie.genre & 98 & 37 \\
movie.gross\_revenue & 34 & 20 \\
movie.language & 207 & 72 \\
movie.location & 21 & 11 \\
movie.name & 3157 & 1030 \\
movie.release\_date & 201 & 70 \\
movie.release\_region & 10 & 6 \\
movie.star\_rating & 1 & 1 \\
movie.subject & 247 & 59 \\
movie.type & 0 & 4 \\
person.name & 280 & 66 \\
person.nationality & 2 & 0 \\
producer.name & 336 & 121 \\
rating.name & 240 & 69 \\

    	\bottomrule
	\end{tabular}
    \caption{Frequency of the concepts in the corpora.}
	\label{tab:frequencies}
\end{table}

The data set used is NL-SPARQL, a corpora of sentences from the movie domain.
For each word, the \ac{POS} and the word stem are provided in addition to the concept tag.
Example of concepts are the actor name or the movie release date.
The aim is to train a model to assign the most probable sequence of concepts to each sentence.

\subsection{NL-SPARQL Corpora}
The corpora is already divided in the train and test sets.
The train set consists of $\input{counts/train.sentences}$ sentences, for a total of $\input{counts/train.tokens}$ tokens.
The test set is approximately a third of the train one and consists of $\input{counts/test.sentences}$ sentences, for a total of $\input{counts/test.tokens}$ tokens.
The corpora contains $\input{counts/corpora.concepts}$ different concepts.
\cref{tab:frequencies} show their frequencies in the train and test sets.
$9$ concepts, for instance \texttt{director.nationality} and \texttt{award.category}, are very rare in the train set, with a frequency smaller then $10$.
The concept \texttt{movie.type} is present only in the test set, so it is very hard for any model to predict it.

\subsection{IOB notation}
The concept tags follow the \ac{IOB} format.
Each concept can spawn on multiple contingent words.
The \texttt{O} tag indicates that the word has not being assigned any concept.
The \texttt{B-} prefix denotes the beginning of a chunk,
the \texttt{I-} prefix indicates a chunk continuation.
Each chunk is delimited by a \texttt{O}, the next \texttt{B-} tag or the end of the sentence.

\subsection{Evaluation}
The performances of the models are evaluated using the \texttt{conlleval.pl}\footnote{\url{http://www.cnts.ua.ac.be/conll2000/chunking/output.html}},
a Perl script originally developed to measure the performance of \ac{POS} tagging the CoNLL-2000 corpora.
The script can handle the \ac{IOB} notation and computes precision, recall and F1 score for each concept and the overall performances for the entire test set.
