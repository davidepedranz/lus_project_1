\documentclass[11pt,a4paper]{article}
\usepackage[hyperref]{acl2017}
\usepackage{times}
\usepackage{latexsym}
\usepackage{url}
\usepackage{acronym}

% use a decent font
\usepackage[T1]{fontenc}

% TODO: cancellare pacchetti non usati!
\usepackage{amsmath}
\usepackage{graphicx}
\usepackage{cite}
\usepackage[caption=false,font=footnotesize]{subfig}
\usepackage[binary-units,per-mode=symbol]{siunitx}
\usepackage{booktabs}
\usepackage{pifont}
\usepackage{microtype}
\usepackage{textcomp}
\usepackage[american]{babel}
\usepackage[noabbrev,capitalise]{cleveref}
\usepackage{xspace}
\usepackage{hyphenat}
\usepackage[draft,inline,nomargin,index]{fixme}
\fxsetup{theme=color}
\usepackage{grffile}
\usepackage{xfrac}
\usepackage{multirow}
\RequirePackage{xstring}
\RequirePackage{xparse}




% Uncomment this line for the final submission
\aclfinalcopy

% list of acronyms
\acrodef{SLU}{Spoken Language Understanding}
\acrodef{WFST}{Weighted Finite State Transducer}
\acrodef{POS}{Part of Speech}
\acrodef{IOB}{Inside, Outside, Beginning}

\title{
  Concept Sequence Tagging for a Movie Domain \\
  Language Understanding Systems, Mid-Term Project
} 

\author{Davide Pedranz \\
  Mat. number 189295 \\
  {\tt davide.pedranz@studenti.unitn.it}
}

\date{21 March 2017}

\begin{document}
\maketitle

\begin{abstract}
Concept Sequence Tagging is usually one of the first steps in building a \ac{SLU}.
The concept extracted can be used to build complex systems, like an automatic flights reservation system.
In this project, we developed a \ac{WFST} to extract concepts for a movie domain.
\end{abstract}

\section{Introduction}
\label{sec:introduction}

Concept tagging can be defined as the extraction of concepts out of a given word sequence,
where a concept represents the smallest unit of meaning that is relevant for a specific task.
The extracted concepts can be used by the following blocks in a \ac{SLU} pipeline to provide more complex functions, for instance, personal vocal assistant or an automatic flights reservation system.

In this report, we will present a \ac{WFST} generative model to extract concepts for a movie domain, build using
the \texttt{OpenFst}\footnote{\url{http://www.openfst.org/twiki/bin/view/FST/WebHome}}
and \texttt{OpenGrm NGram}\footnote{\url{http://www.openfst.org/twiki/bin/view/GRM/NGramLibrary}} libraries.
We will briefly describe the used data set.
Then, we will present the generative model.
Finally, we will discuss the \ac{WFST} implementation and the obtained performances.

\section{Data Set}
\label{sec:dataset}

\begin{table}[t!]
	\centering
    \begin{tabular}{ l l l }
    	\toprule
    		\multicolumn{1}{l}{concept} & \multicolumn{1}{l}{train} & \multicolumn{1}{c}{test} \\
    	\midrule
            actor.name & 437 & 157 \\
actor.nationality & 6 & 1 \\
actor.type & 3 & 2 \\
award.category & 1 & 4 \\
award.ceremony & 13 & 7 \\
character.name & 97 & 21 \\
country.name & 212 & 67 \\
director.name & 455 & 156 \\
director.nationality & 2 & 1 \\
movie.description & 2 & 0 \\
movie.genre & 98 & 37 \\
movie.gross\_revenue & 34 & 20 \\
movie.language & 207 & 72 \\
movie.location & 21 & 11 \\
movie.name & 3157 & 1030 \\
movie.release\_date & 201 & 70 \\
movie.release\_region & 10 & 6 \\
movie.star\_rating & 1 & 1 \\
movie.subject & 247 & 59 \\
movie.type & 0 & 4 \\
person.name & 280 & 66 \\
person.nationality & 2 & 0 \\
producer.name & 336 & 121 \\
rating.name & 240 & 69 \\

    	\bottomrule
	\end{tabular}
    \caption{Frequency of the concepts in the corpora.}
	\label{tab:frequencies}
\end{table}

The data set used is NL-SPARQL, a corpora of sentences from the movie domain.
For each word, the \ac{POS} and the word stem are provided in addition to the concept tag.
Example of concepts are the actor name or the movie release date.
The aim is to train a model to assign the most probable sequence of concepts to each sentence.

\subsection{NL-SPARQL Corpora}
The corpora is already divided in the train and test sets.
The train set consists of $\input{counts/train.sentences}$ sentences, for a total of $\input{counts/train.tokens}$ tokens.
The test set is approximately a third of the train one and consists of $\input{counts/test.sentences}$ sentences, for a total of $\input{counts/test.tokens}$ tokens.
The corpora contains $\input{counts/corpora.concepts}$ different concepts.
\cref{tab:frequencies} show their frequencies in the train and test sets.
$9$ concepts, for instance \texttt{director.nationality} and \texttt{award.category}, are very rare in the train set, with a frequency smaller then $10$.
The concept \texttt{movie.type} is present only in the test set, so it is very hard for any model to predict it.

\subsection{IOB notation}
The concept tags follow the \ac{IOB} format.
Each concept can spawn on multiple contingent words.
The \texttt{O} tag indicates that the word has not being assigned any concept.
The \texttt{B-} prefix denotes the beginning of a chunk,
the \texttt{I-} prefix indicates a chunk continuation.
Each chunk is delimited by a \texttt{O}, the next \texttt{B-} tag or the end of the sentence.

\subsection{Evaluation}
The performances of the models are evaluated using the \texttt{conlleval.pl}\footnote{\url{http://www.cnts.ua.ac.be/conll2000/chunking/output.html}},
a Perl script originally developed to measure the performance of \ac{POS} tagging the CoNLL-2000 corpora.
The script can handle the \ac{IOB} notation and computes precision, recall and F1 score for each concept and the overall performances for the entire test set.


% scaletta
% 2 -> dataset, train len, test len, # tags, IOB notation, evaluation script
% 3 -> modello teorico, vedi slide russo
% 4 -> modello 1
% 5 -> migliormento, modello 2
% 6 -> conclusioni

\end{document}
