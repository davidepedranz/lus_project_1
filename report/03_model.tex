\section{Model}
\label{sec:model}

The model computes the sequence of tags $t_1, ..., t_n$ with the highest probability given the observed sequence of words 
$w_1, ..., w_n$:
\begin{equation*}
    t_1, ..., t_n = \argmax_{t_1, ..., t_n} P(t_1^n | w_1^n).
\end{equation*}

Applying the Bayes' Law, we obtain:
\begin{equation*}
    t_1, ..., t_n = \argmax_{t_1, ..., t_n} \frac{P(w_1^n | t_1^n) P(t_1^n)}{P(w_1^n)}.
\end{equation*}

Since we maximize over $t_1, ..., t_n$, we can drop the denominator and obtain:
\begin{equation}
    t_1, ..., t_n = \argmax_{t_1, ..., t_n} P(w_1^n | t_1^n) P(t_1^n).
\end{equation}

For this model, we assume that the probability of a word only depends on its own tag,
not on the tags of the other words in sentence.
Thus, we have that:
\begin{equation}
    P(w_1^n | t_1^n) \approx \prod_{i=1}^n P(w_1 | t_1) 
\end{equation}

Each $P(w_i | t_i)$ can be easily computed from the training data as:
\begin{equation}
    P(w_i | t_i) = \frac{C(t_i, w_i)}{C(t_i)},
\end{equation}
where $C(t_i, w_i)$ denotes the number of times that word $w_i$ has tag $t_i$ 
and $C(t_i)$ denotes the number of occurences of  ag $t_i$ in the training set.

The term $P(t_1^n)$ represent the probability of a tag given all its predecessor in the sentence.
This can be computed, but suffers from sparness problems.
To compute the term, we introduce a Markov assumption: the probability of a tag in the sentence
depends only on the last $m$ tags.
For $m=1$ (bigrams), we have:
\begin{equation}
    P(t_1, ..., t_n) = \prod_{i=1}^n P(t_i | t_{i-1}),
\end{equation}
where $t_0$ indicates the beginning of the sentence.

Thus, our final model (for $m=1$) can be computed as:
\begin{equation}
    t_1^n = \argmax_{t_1, ..., t_n} \prod_{i=1}^n P(w_1 | t_1) \prod_{i=1}^n P(t_i | t_{i-1}).
\end{equation}

It is possible to compute the model for different values of $m$.

%%% m=0 baseline!

This model can be realized as the compositions of 2 \ac{WFST}:
a ........
